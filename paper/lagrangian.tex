%!TEX TS-program = latex
%\documentclass[10pt,a4paper,oneside]{article}
\documentclass[]{svjour3}
%%\documentclass[12pt,a4paper,oneside]{amsart}
%%\usepackage{amsfonts,mathrsfs}
%%\usepackage{amsmath, amsthm, amssymb}
\usepackage{amsmath,mathrsfs}
%  \newcommand{\field}[1]{\mathbb{#1}}
%\usepackage{amsfonts}
%%%\usepackage{sectsty}

\usepackage[dvips]{graphicx}
\usepackage{algorithm}
\usepackage{algorithmic}
%%\usepackage{lineno}
\usepackage[hang]{subfigure}
\usepackage{mathptmx}

%\renewcommand \thesection{\Roman{section}.}
%\renewcommand \thesubsection{\Roman{section}.\Alph{subsection}.}

%\font\chapterfontsans=cmss12 scaled \magstep 2

%\usepackage{amsmath}
%\usepackage{chicago}
\usepackage{natbib}

\usepackage{bm}
%\oddsidemargin=11mm
\oddsidemargin=3mm
\textwidth=16.0cm
\topmargin=-1.0cm
\textheight=250mm
%\textheight=606pt
%\textwidth =15.cm
%\headheight=15pt
\hoffset=0.0cm
\parindent=6mm
%%\setlength{\parskip}{0.2cm}
%%\setlength{\parskip}{0.0cm}
%%%\renewcommand{\baselinestretch}{1.24}  % 1.5 Line Spacing
%\renewcommand{\baselinestretch}{1.66}  % 2.0 Line Spacing

\def\ba{\begin{eqnarray*}}
\def\ea{\end{eqnarray*}}
\def\ban{\begin{eqnarray}}
\def\ean{\end{eqnarray}}

\def\beq{\begin{equation}}
\def\eeq{\end{equation}}

\def\s{\sigma}
\def\t{\tau}
\def\e{\epsilon}

\def\CL{\mathscr{L}}

%\newtheorem{theorem}{\fontB Theorem}
%\newtheorem{definition}{Definition}
%\newtheorem{cororally}{Cororally}
%\newtheorem{lemma}{Lemma}

\newcommand\helvetica{\usefont{T1}{phv}{m}{n}}
\DeclareTextFontCommand{\texthelv}{\helvetica}
\newcommand\fontA{\usefont{T1}{pag}{db}{n}}
\DeclareTextFontCommand{\textfontA}{\fontA}
\newcommand\fontB{\usefont{T1}{phvv}{b}{n}}
\DeclareTextFontCommand{\textfontB}{\fontB}
\newcommand\fontC{\usefont{T1}{phvv}{m}{n}}
\DeclareTextFontCommand{\textfontC}{\fontC}

%\usepackage[T1]{fontenc}

%%\allsectionsfont{\fontB}
%%%%%%%%%%%%%%%%%%%%%%%%%%%%%%%%%%%%%%%%%%%%%%%%%%%%%%%%%%%%%%%%%%%%%%%%%%%
%%%%%%%%%%%%%%%%%%%%%%%%%%%%%%%%%%%%%%%%%%%%%%%%%%%%%%%%%%%%%%%%%%%%%%%%%%%
%%%%%%%%%%%%%%%%%%%%%%%%%%%%%%%%%%%%%%%%%%%%%%%%%%%%%%%%%%%%%%%%%%%%%%%%%%%
\begin{document}

\title{Lagrangian Augmented}
\author{ Louis Moresi \and Mirko Veli\'c}
\institute{ School of Mathematical Sciences,\\ Building 28, Monash University, \\
Clayton, Victoria 3800, Australia. \\ 
\email{louis.moresi@monash.edu}}
%%%%%%%%%%%%%%%%%%%%%%%%%%%%%%

%%%%%%%%%%%%%%%%%%%%%%%%%%%%%%

\maketitle

%%\linenumbers

%%\begin{quote}
\abstract{The Augmented Lagrangian method.
\keywords{Augmented Lagrangian method \and Penalty method}
}
%%\end{quote}


\section{Introduction}

\section{Augmented Lagrangian}
The Stokes' saddle point problem,
\beq
   \left[\begin{array}{cc}
       K   & G \rule{0pt}{13pt}\\
       G^T & 0 \rule{0pt}{13pt}
   \end{array}\right]
   \left[
\begin{array}{c}
u  \rule{0pt}{13pt}\\
p   \rule{0pt}{13pt}
\end{array}
\right] 
=
\left[\begin{array}{c}
f  \rule{0pt}{13pt}\\
h   \rule{0pt}{13pt}
\end{array}
\right]
\label{stokes}
\eeq
may be regarded as the result of the Lagrange multiplier method applied to solving
\beq
   K u = f
\eeq
subject to the constraint,
\beq
  G^T u  = h,
\eeq
where $p$ is a vector of Langrange multipliers. i.e.,
The Stokes' system gives the solution which is the extremum of
\beq
    \phi = \frac{1}{2}u^T K u - u^T f + p^T(G^T u -h).
\eeq

The augmented lagrangian method may be obtained as the extremum of
\beq
    \phi = \frac{1}{2}u^T K u - u^T f + p^T(G^T u -h) + \frac{\lambda}{2}(G^T u -h)^2,\label{al}
\eeq
where a term with a penalty factor has beed added (augmented).

Differentiating (\ref{al}) gives,
\beq
   K u - f + G p + \lambda G(G^T u - h) = 0
\eeq
\beq
  G^T u -h =0.
\eeq

\beq
   \left[\begin{array}{cc}
       K + \lambda G G^T  & G \rule{0pt}{13pt}\\
       G^T & 0 \rule{0pt}{13pt}
   \end{array}\right]
   \left[
\begin{array}{c}
u  \rule{0pt}{13pt}\\
p   \rule{0pt}{13pt}
\end{array}
\right] 
=
\left[\begin{array}{c}
f + \lambda G h \rule{0pt}{13pt}\\
h   \rule{0pt}{13pt}
\end{array}
\right]
\label{mag}
\eeq

\section{Scaling}
A consistent scaling of (\ref{mag}) is
\beq
     (D K D^T)(D^{-T} u) - D f + D G p + \lambda D G(G^T D^T D^{-T} u - h) = 0
\eeq

\beq
   \left[\begin{array}{cc}
       D K D^T + \lambda D G G^T D^T  & D G \rule{0pt}{13pt}\\
       G^T D^T & 0 \rule{0pt}{13pt}
   \end{array}\right]
   \left[
   \begin{array}{c}
     D^{-T}u  \rule{0pt}{13pt}\\
     p   \rule{0pt}{13pt}
   \end{array}
   \right] 
   =
   \left[\begin{array}{c}
       D f + \lambda D G h \rule{0pt}{13pt}\\
       h   \rule{0pt}{13pt}
     \end{array}
     \right].
   \label{scaledmag}
\eeq


The Schur compliment of (\ref{scaledmag}) is
\beq
    G^T D^T (D K D^T + \lambda D G G^T D^T)^{-1} D G = G^T D^T D^{-T} (K + \lambda G G^T)^{-1} D^{-1} D G =  G^T (K + \lambda G G^T)^{-1} G.
\eeq
and is identical to the compliment of (\ref{stokes}).

\section{Boundary Conditions + Method}
It is easy to show that the application of boundary conditions and the augmented lagrangian method to (\ref{stokes})
commute.

Decompose into reduced system and boundary condition terms.
\beq
   \left[\begin{array}{ccc}
       K_R & K_B    & G_R \rule{0pt}{13pt}\\
       K_C & K_{BC}  & G_B \rule{0pt}{13pt}\\
       G_R^T & G_B^T & 0 \rule{0pt}{13pt}
   \end{array}\right]
   \left[\begin{array}{c}
u_R  \rule{0pt}{13pt}\\
u_B  \rule{0pt}{13pt}\\
p   \rule{0pt}{13pt}
\end{array}
\right] 
=
\left[\begin{array}{c}
f_R \rule{0pt}{13pt}\\
f_B \rule{0pt}{13pt}\\
h   \rule{0pt}{13pt}
\end{array}
\right]
\eeq
This reduces to, after moving boundary condition terms to RHS
\beq
   \left[\begin{array}{cc}
       K_R   & G_R \rule{0pt}{13pt}\\
       G_R^T & 0 \rule{0pt}{13pt}
   \end{array}\right]
   \left[\begin{array}{c}
u_R  \rule{0pt}{13pt}\\
p   \rule{0pt}{13pt}
\end{array}
\right] 
=
\left[\begin{array}{c}
f_R -K_B u_B\rule{0pt}{13pt}\\
h   -G_B^T u_B\rule{0pt}{13pt}
\end{array}
\right]
\eeq
\section{Epilogue}

\begin{acknowledgements}

 ....

\end{acknowledgements}

%% this forces all figs so far to be printed. Now hopefully they won't appear in the middle of the ref list.
%%\cleardoublepage
%%\bibliographystyle{plain}
%%\bibliography{bib}



%%%%%%%%%%%%%%%%%%%%%%%%%%%%%%%%%%%%%%%%%%%%%%%%%%%%%%%%%%%%%%%%%%%%%%%%%%%
%%%%%%%%%%%%%%%%%%%%%%%%%%%%%%%%%%%%%%%%%%%%%%%%%%%%%%%%%%%%%%%%%%%%%%%%%%%
%%%%%%%%%%%%%%%%%%%%%%%%%%%%%%%%%%%%%%%%%%%%%%%%%%%%%%%%%%%%%%%%%%%%%%%%%%%
\end{document}
%%%%%%%%%%%%%%%%%%%%%%%%%%%%%%%%%%%%%%%%%%%%%%%%%%%%%%%%%%%%%%%%%%%%%%%%%%%
%%%%%%%%%%%%%%%%%%%%%%%%%%%%%%%%%%%%%%%%%%%%%%%%%%%%%%%%%%%%%%%%%%%%%%%%%%%
%%%%%%%%%%%%%%%%%%%%%%%%%%%%%%%%%%%%%%%%%%%%%%%%%%%%%%%%%%%%%%%%%%%%%%%%%%%


