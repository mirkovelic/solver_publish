%!TEX TS-program = latex
%\documentclass[10pt,a4paper,oneside]{article}
\documentclass[]{svjour3}
%%\documentclass[12pt,a4paper,oneside]{amsart}
%%\usepackage{amsfonts,mathrsfs}
%%\usepackage{amsmath, amsthm, amssymb}
\usepackage{amsmath,mathrsfs}
%  \newcommand{\field}[1]{\mathbb{#1}}
%\usepackage{amsfonts}
%%%\usepackage{sectsty}

\usepackage[dvips]{graphicx}
\usepackage{algorithm}
\usepackage{algorithmic}
%%\usepackage{lineno}
\usepackage[hang]{subfigure}
\usepackage{mathptmx}

%\renewcommand \thesection{\Roman{section}.}
%\renewcommand \thesubsection{\Roman{section}.\Alph{subsection}.}

%\font\chapterfontsans=cmss12 scaled \magstep 2

%\usepackage{amsmath}
%\usepackage{chicago}
\usepackage{natbib}

\usepackage{bm}
%\oddsidemargin=11mm
\oddsidemargin=3mm
\textwidth=16.0cm
\topmargin=-1.0cm
\textheight=250mm
%\textheight=606pt
%\textwidth =15.cm
%\headheight=15pt
\hoffset=0.0cm
\parindent=6mm
%%\setlength{\parskip}{0.2cm}
%%\setlength{\parskip}{0.0cm}
%%%\renewcommand{\baselinestretch}{1.24}  % 1.5 Line Spacing
%\renewcommand{\baselinestretch}{1.66}  % 2.0 Line Spacing

\def\ba{\begin{eqnarray*}}
\def\ea{\end{eqnarray*}}
\def\ban{\begin{eqnarray}}
\def\ean{\end{eqnarray}}

\def\beq{\begin{equation}}
\def\eeq{\end{equation}}

\def\s{\sigma}
\def\t{\tau}
\def\e{\epsilon}

\def\CL{\mathscr{L}}

%\newtheorem{theorem}{\fontB Theorem}
%\newtheorem{definition}{Definition}
%\newtheorem{cororally}{Cororally}
%\newtheorem{lemma}{Lemma}

\newcommand\helvetica{\usefont{T1}{phv}{m}{n}}
\DeclareTextFontCommand{\texthelv}{\helvetica}
\newcommand\fontA{\usefont{T1}{pag}{db}{n}}
\DeclareTextFontCommand{\textfontA}{\fontA}
\newcommand\fontB{\usefont{T1}{phvv}{b}{n}}
\DeclareTextFontCommand{\textfontB}{\fontB}
\newcommand\fontC{\usefont{T1}{phvv}{m}{n}}
\DeclareTextFontCommand{\textfontC}{\fontC}

%\usepackage[T1]{fontenc}

%%\allsectionsfont{\fontB}
%%%%%%%%%%%%%%%%%%%%%%%%%%%%%%%%%%%%%%%%%%%%%%%%%%%%%%%%%%%%%%%%%%%%%%%%%%%
%%%%%%%%%%%%%%%%%%%%%%%%%%%%%%%%%%%%%%%%%%%%%%%%%%%%%%%%%%%%%%%%%%%%%%%%%%%
%%%%%%%%%%%%%%%%%%%%%%%%%%%%%%%%%%%%%%%%%%%%%%%%%%%%%%%%%%%%%%%%%%%%%%%%%%%
\begin{document}

\title{Lagrangian Augmented}
\author{ Louis Moresi \and Mirko Veli\'c}
\institute{ School of Mathematical Sciences,\\ Building 28, Monash University, \\
Clayton, Victoria 3800, Australia. \\ 
\email{louis.moresi@monash.edu}}
%%%%%%%%%%%%%%%%%%%%%%%%%%%%%%

%%%%%%%%%%%%%%%%%%%%%%%%%%%%%%

\maketitle

%%\linenumbers

%%\begin{quote}
\abstract{The Augmented Lagrangian method.
\keywords{Augmented Lagrangian method \and Penalty method}
}
%%\end{quote}


\section{Introduction}

Incompressible Stokes flow solver is the main building block for codes designed to model plate-tectonics: the coupled deformation
of the Earth's lithosphere, asthenosphere and mantle.  Implicit, elliptic ... 

On the planetary scale, the mantle is compressible, but the appropriate approximation is to assume a decoupling
of the dynamic, elastic compressible effects (e.g. seismic wave propagation) and the slow fluid-flow of material
with respect to a compressible reference state. This latter formulation poses identical numerical difficulties to
the incompressible problem.

FEM is commonly used because it provided a robust way to formulate complicated constitutive laws, conform meshes to
the emergent, complicated geometry of the domain, and consistent mechanisms to handle steps in material properties.
However, the application of the incompressibility constraint provides a significant challenge.

In many respects, this challenge is a recent one. When I was a lad, we always used the penalty method [hughes ...]
combined with a direct solver to obtain accurate, close-to-incompressible flow solutions with high accuracy [moresi et al].
However, as high resolution, 3D, and global-scale computations have become routine, the poor-scaling of direct
solvers with increasing problem size has led to the widespread adoption of iterative methods in this field, 
particularly those based on multigrid. 

The matrices arising from the penalty method are typically very ill-conditioned and most iterative methods 
are poorly suited to such systems. Instead, one of the most popular approaches currently in use is to 
apply a Schur-complement reduction to the block system describing the constrained flow and to solve this
system with one of the standard Krylov-subspace iterative methods [Fortin, Wathen and all that lot ... need to
check the book to see what they cover of this already]. This
results in a hierarchical solution approach with two requirements: an efficient, robust \textit{inner solver} 
for the unconstrained problem; and an effective preconditioner for the Schur complement system [may et al, 2009].
In addition to standard preconditioning, it is also possible to use a hybrid \textit{Augmented Lagrangian} method
to improve the convergence of the krylov sub-space method. This approach includes a term into the 
variational problem of the \textit{inner solve} which penalises against violating the constraint equation, 
but with a penalty parameter which small enough to keep the condition number of the inner problem within the 
domain of convergence of the chosen solver. With a sufficiently high penalty and robust inner solve,
this approach is equivalent to the usual penalty method with a back-substitution for the pressure equation.
[[Um ... so what !!]]

Our goal in this paper is to examine the balance between 


\section{Target problems}

All are dominated by rheological variations. Analytic solutions exist for some cases.
They mimic the typical problems we encounter in models of plate tectonics

The following figures illustrate the condition numbers of the Stokes stiffness matrix ($K$) as well as 
those of the Schur compliment matrix ($S$).

The condition numbers are calculated by taking the singular value decomposition of the matrices.
In the case of the Schur matrix we have divided the largest singular value by the second smallest as the minimal value is always zero due to the constant null-space associated with the pressure.

This gives an effective condition number that reflects the response of iterative methods to this matrix.
This can be seen in the second set of figures showing the total inner iterations of the Schur compliment (Pressure) solve for the
Stokes problem.

The effect of the penalty on all of these examples appears to be fairly consistent.

The Q1-P0 element examples show that the augmented lagrangian method improves the convergence for smallish penalty values while
it appears to have no negative impact for the same values for the higher order mesh examples.

Whilst the speed-up or advantage does not appear dramatic for these simple examples.
It is important that it is not a disadvantage for simpler models.
The augmented lagrangian method will often allow the more extensive models to either converge where they previously would not or provide a dramatic improvement in convergence.
(examples?)

In a nutshell the method improves the condition number of the Schur system and therefore hopefully it's convergence properties
whilst degrading somewhat the properties of the matrix associated with the inner (velocity) solves. But we expect to offset this
with tools like multigrid which are available for this part of the solve hierarchy.

\section{Augmented Lagrangian}
The Stokes' saddle point problem,
\beq
   \left[\begin{array}{cc}
       K   & G \rule{0pt}{13pt}\\
       G^T & 0 \rule{0pt}{13pt}
   \end{array}\right]
   \left[
\begin{array}{c}
u  \rule{0pt}{13pt}\\
p   \rule{0pt}{13pt}
\end{array}
\right] 
=
\left[\begin{array}{c}
f  \rule{0pt}{13pt}\\
h   \rule{0pt}{13pt}
\end{array}
\right]
\label{stokes}
\eeq
may be regarded as the result of the Lagrange multiplier method applied to solving
\beq
   K u = f
\eeq
subject to the constraint,
\beq
  G^T u  = h,
\eeq
where $p$ is a vector of Langrange multipliers. i.e.,
The Stokes' system gives the solution which is the extremum of
\beq
    \phi = \frac{1}{2}u^T K u - u^T f + p^T(G^T u -h).
\eeq

The augmented lagrangian method may be obtained as the extremum of
\beq
    \phi = \frac{1}{2}u^T K u - u^T f + p^T(G^T u -h) + \frac{\lambda}{2}(G^T u -h)^2,\label{al}
\eeq
where a term with a penalty factor has beed added (augmented).

Differentiating (\ref{al}) gives,
\beq
   K u - f + G p + \lambda G(G^T u - h) = 0
\eeq
\beq
  G^T u -h =0.
\eeq

\beq
   \left[\begin{array}{cc}
       K + \lambda G G^T  & G \rule{0pt}{13pt}\\
       G^T & 0 \rule{0pt}{13pt}
   \end{array}\right]
   \left[
\begin{array}{c}
u  \rule{0pt}{13pt}\\
p   \rule{0pt}{13pt}
\end{array}
\right] 
=
\left[\begin{array}{c}
f + \lambda G h \rule{0pt}{13pt}\\
h   \rule{0pt}{13pt}
\end{array}
\right]
\label{mag}
\eeq

\section{Scaling}
A consistent scaling of (\ref{mag}) is
\beq
     (D K D^T)(D^{-T} u) - D f + D G p + \lambda D G(G^T D^T D^{-T} u - h) = 0
\eeq

\beq
   \left[\begin{array}{cc}
       D K D^T + \lambda D G G^T D^T  & D G \rule{0pt}{13pt}\\
       G^T D^T & 0 \rule{0pt}{13pt}
   \end{array}\right]
   \left[
   \begin{array}{c}
     D^{-T}u  \rule{0pt}{13pt}\\
     p   \rule{0pt}{13pt}
   \end{array}
   \right] 
   =
   \left[\begin{array}{c}
       D f + \lambda D G h \rule{0pt}{13pt}\\
       h   \rule{0pt}{13pt}
     \end{array}
     \right].
   \label{scaledmag}
\eeq


The Schur compliment of (\ref{scaledmag}) is
\beq
    G^T D^T (D K D^T + \lambda D G G^T D^T)^{-1} D G = G^T D^T D^{-T} (K + \lambda G G^T)^{-1} D^{-1} D G =  G^T (K + \lambda G G^T)^{-1} G.
\eeq
and is identical to the compliment of (\ref{stokes}).

\section{Boundary Conditions + Method}
It is easy to show that the application of boundary conditions and the augmented lagrangian method to (\ref{stokes})
commute.

Decompose into reduced system and boundary condition terms.
\beq
   \left[\begin{array}{ccc}
       K_R & K_B    & G_R \rule{0pt}{13pt}\\
       K_C & K_{BC}  & G_B \rule{0pt}{13pt}\\
       G_R^T & G_B^T & 0 \rule{0pt}{13pt}
   \end{array}\right]
   \left[\begin{array}{c}
u_R  \rule{0pt}{13pt}\\
u_B  \rule{0pt}{13pt}\\
p   \rule{0pt}{13pt}
\end{array}
\right] 
=
\left[\begin{array}{c}
f_R \rule{0pt}{13pt}\\
f_B \rule{0pt}{13pt}\\
h   \rule{0pt}{13pt}
\end{array}
\right]
\eeq
This reduces to, after moving boundary condition terms to RHS
\beq
   \left[\begin{array}{cc}
       K_R   & G_R \rule{0pt}{13pt}\\
       G_R^T & 0 \rule{0pt}{13pt}
   \end{array}\right]
   \left[\begin{array}{c}
u_R  \rule{0pt}{13pt}\\
p   \rule{0pt}{13pt}
\end{array}
\right] 
=
\left[\begin{array}{c}
f_R -K_B u_B\rule{0pt}{13pt}\\
h   -G_B^T u_B\rule{0pt}{13pt}
\end{array}
\right]
\eeq

\subsection{Vertical layerings}
The solkx solution has a viscosity variation that is exponential.

%%\includegraphics[width=\textwidth]{fig/condNumberssolkxq2q1__no_scale.pdf}

%%\includegraphics[width=\textwidth]{fig/workPtimesolkxq2q1_totalWork_no_scale.pdf}

\subsection{Lateral variations}
The solcx solution has a step lateral viscosity jump mid-way along a $1x1$ box.

%\includegraphics[width=\textwidth]{fig/condNumberssolcxq2q1__no_scale.pdf}

%\includegraphics[width=\textwidth]{fig/workPtimesolcxq2q1_totalWork_no_scale.pdf}

\subsection{Sinking blobs in 2D and 3D}
A sinking viscous blob.

%\includegraphics[width=\textwidth]{fig/condNumberssolsinkerq2q1__no_scale.pdf}

%\includegraphics[width=\textwidth]{fig/workPtimesolsinkerq2q1_totalWork_no_scale.pdf}


\section{Momentum balance}

The momentum balance equation is:
\begin{equation}
	\tau_{ij,j} - p_{,i} = f_i
\end{equation}
Substituting the constitutive equation (\ref{eq:VEConstStiffnessForm}) 
\begin{equation}
	\frac{\partial }{\partial x_j} \left( 
		2 \eta_{\rm eff} D^t_{ij}(\mathbf{x}^t_p) +
		\frac{\eta_{\rm eff}}{\mu \Delta t_e}
		 {\tau}_{ij}^t(\Delta t_e, \mathbf{x}^t_p, \mathbf{u}) ) \right)
		- p_{,i} = f_{,i}
\end{equation}
or (\ref{eq:VEConstStiffnessForm2}) 
\begin{equation}
	\frac{\partial }{\partial x_j} \left( 
		2 \eta_{\rm eff} D^t_{ij}(\mathbf{x}^t_p) +
		\frac{\eta_{\rm eff}}{\mu \Delta t_e}
		 	\mathcal{T}^{x^t_p}_{\Delta t_e} \left(
			 \tau_{ij}^{t-\Delta t_e}(\mathbf{x}_p^{t-\Delta t_e}) \right) \right)
		- p_{,i} = f_{,i}
\end{equation}


\section{Epilogue}

\begin{acknowledgements}

 ....

\end{acknowledgements}

%% this forces all figs so far to be printed. Now hopefully they won't appear in the middle of the ref list.
%%\cleardoublepage
%%\bibliographystyle{plain}
%%\bibliography{bib}



%%%%%%%%%%%%%%%%%%%%%%%%%%%%%%%%%%%%%%%%%%%%%%%%%%%%%%%%%%%%%%%%%%%%%%%%%%%
%%%%%%%%%%%%%%%%%%%%%%%%%%%%%%%%%%%%%%%%%%%%%%%%%%%%%%%%%%%%%%%%%%%%%%%%%%%
%%%%%%%%%%%%%%%%%%%%%%%%%%%%%%%%%%%%%%%%%%%%%%%%%%%%%%%%%%%%%%%%%%%%%%%%%%%
\end{document}
%%%%%%%%%%%%%%%%%%%%%%%%%%%%%%%%%%%%%%%%%%%%%%%%%%%%%%%%%%%%%%%%%%%%%%%%%%%
%%%%%%%%%%%%%%%%%%%%%%%%%%%%%%%%%%%%%%%%%%%%%%%%%%%%%%%%%%%%%%%%%%%%%%%%%%%
%%%%%%%%%%%%%%%%%%%%%%%%%%%%%%%%%%%%%%%%%%%%%%%%%%%%%%%%%%%%%%%%%%%%%%%%%%%


